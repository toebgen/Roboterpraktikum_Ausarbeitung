\chapter{Analysenmethoden zur Bestimmung der Reaktionsvariablen }

\section{Gaschromatographie}

Die Bestimmung des Wasserstoffanteils mit dem GC HP-5880 A Series erfolgt durch eine Temperaturrampe. Zuerst wird die Temperatur fünf Minuten lang bei $50 ^{\circ}C$ gehalten. Danach wird bis $130 ^{\circ}C$ ($15 ^{\circ}C/min$) aufgeheizt. Der GC besaß eine gepackte Trennsäule (Porapak Q). Als Detektor stand ein Wärmeleitfähigkeitsdetektor (WLD) zu Verfügung. 

Der GC HP 6890 ist mit zwei Säulen ausgestattet, eine für die Bestimmung von Wasserstoff (80/100 Hayesep Q, 2 m lang)  mit Stickstoff als Trägergas, und eine andere Säule (60/80 Molekularsieb, 4 m lang) für CO2, Stickstoff, Sauerstoff und Kohlenwasserstoffe ($< C4$), in der Helium als Trägergas benutzt wird. Für die Konzentrationsmessung wurde zwei Detektoren verwendet: ein Wärmeleitfähigkeitsdetektor (WLD) und ein Flammenionisationsdetektor (FID), die in Serie geschaltet waren. 

In dem WLD wird die Wärmeleitfähigkeit des Trägergases (Stickstoff) mit der der gaschromatographisch getrennten Substanzen verglichen. Der Vergleich erfolgt an einer Wheatstoneschen Brücken. Solange nur Trägergas durch die Wheatstoneschen Brücken fließt, wird die Wärme vollständig abgeführt. Wenn andere Substanzen mit geringe Leitfähigkeit in die Messzelle gelangen, entsteht ein Wärmestau am Hitzdraht (Platin), der den elektrischen Wiederstand erhöht. 

Für die anderen Komponenten in der Gasphase wird ein FID benutzt, der für Verbindungen mit C-H oder C-C Bindungen eine deutlich höhere Empfindlichkeit besitzt. Die Verbrennung der Komponenten läuft über die Bildung von Radikalen, durch die die Ionisation erhöht wird. 

Die Messung erfolgt während 31 Minuten. Die Temperatur wird von $60 ^{\circ}C$ auf $220 ^{\circ}C$ erhöht. Zuerst wird nach der Probeeinspritzung die Temperatur bei $60 ^{\circ}C$ zwei Minuten lang gehalten. Dann erfolgt eine Aufheizungsrate von $5 ^{\circ}C/min$  bis eine Temperatur von $160 ^{\circ}C$ erreicht wird. Anschließend steigt die Temperatur bis $220 ^{\circ}C$ mit einer Aufheizungsrate von $15 ^{\circ}C/min$. Bei $220 ^{\circ}C$ wird die Temperatur während  5 min konstant gehalten. 

Die Gasproben werden mit Gasspritzen aus zwei Probenehmern  gezogen.  Ein erster Probenehmer befand sich direkt an der Stelle der Phasentrennung, der zweite war hinter dem Abwasserbehälter angebracht. 
Zur Kalibrierung der Gas-Chromatographen wurden täglich Gasproben mit bekannter Zusammensetzung eingespritzt. Da die Gas-Chromatographen von mehreren Arbeitsgruppen bedient wurden, war eine Ausheizung des GC während mehreren Stunden nötig, die jedes zweites Wochenende durchgeführt wurde.  Die Versuchsplanung hängte davon ab, dass die Gasproben gemessen werden konnten.

\section{Analyse zur Bestimmung des Kohlenstoffgehalts im Abwasser}

TOC-Messgerät (DC-190 Rosemount-Dohrman) 

Das Messprinzip besteht aus einer hochtemperatur-katalysierten Verbrennung der Abwasserprobe. Die Verbrennung findet in einem Quarz-Rohr statt, das  gepackt mit dem Katalysator Pt/Al2O3 ist. Die Ofentemperatur  beträgt $800 ^{\circ}C$. Durch die katalysierte unterstützte Oxidierung mit reinem Sauerstoff ($250 ml/min$), werden die Abwasserproben vollständig zu CO2 und Wasser umgewandelt. Das entstehende CO2 wird von dem Wasser in einem Kondensator-Gas-Flüssig-Trennungssystem (Kupfer-Zinn-Adsorber) gereinigt. Die Halogene werden von dem Gasprodukt getrennt. Die reine CO2-Menge wird anhand eines Infrarotdetektors ermittelt. Anorganischer Kohlenstoff (IC) wird mit einem IC  Reaktor bestimmt, der Phosphorsäure 20 \% enthält. In diesem säuerlichen Medium werden die  anorganischen Carbonate der Abwasserprobe zu CO2 umgewandelt. Das entstehende Gas wird zum Trennungssystem geleitet und mit dem Infrarotdetektor gemessen. Die Genauigkeit dieses Messprinzips lag um etwa 2 \%. 

Küvetten-Test LCK 381 (Dr. Lange GmbH, Düsseldorf, D, Messbereich 60-735 mg/l)

Die Proben müssen verdünnt werden, wenn der Messbereich überschritten werden könnte.  Zur Bestimmung des Gesamtkohlenstoff-Gehaltes (TC) wurden $200 \mu l$ der verdünnten Abwasserprobe in die mit dem Aufschlussreagenz (Natriumperoxidsulfat) gefüllte TC-Küvette (KD 381 B) pipettiert. Nach dem Verschließen mit Originaldeckel und mehrmaligem Umschwenken wurde die TC-Küvette mit einer vorbereiteten Indikatorküvette (LCK 380/381) durch einen Membran-Doppel-deckel verbunden. Gleichzeitig wurde die TiC-Küvette (KE 381 B) mit $1 ml$ der Probe befüllt. Analog zur Aufarbeitung der TC-Bestimmung wurde die TiC-Küvette mit dem Originaldeckel verschlossen und mehrfach geschüttelt. Im Anschluß daran wurde die TiC-Küvette mit der entsprechenden Indikatorküvette verbunden. Beide Küvettenkombinationen wurden gleichzeitig in einem vorgeheizten Thermostaten LT 100 (Dr. Lange GmbH) 2 h bei $100^{\circ}C$ erwärmt. Während der Erwärmung reagieren die Abwasserproben mit dem Reagenz und es entsteht CO2, das den Indikator verfärbt. Nach dem Abkühlen auf Raumtemperatur wurden die Indikatorküvetten mit dem Spektralphotometer CADAS-200 (Dr. Lange GmbH) gemessen. Die Indikatorküvetten werden außen gut gesäubert und in den Photometer eingesetzt. Das Barcode-Ettiket wird von der CADAS-200 gelesen, dadurch werden die entsprechenden Wellenlängen für die Durchführung der Messung eingestellt.   

