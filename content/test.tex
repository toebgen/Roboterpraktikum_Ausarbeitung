
\chapter{Test auf realem Roboter}
\authorsection{\editorjulian , \editortobias}
\todo[inline]{Steuerungs-Gruppe: Schreiben}

\section{Testumgebung}

\begin{itemize}
	\item Aufgebaute Umgebung
	\begin{itemize}
		\item FZI, oberes Stockwerk
		\item Abgegrenzter Bereich
		\item Karte erstellen zur Lokalisierung
	\end{itemize}
	\item Geplantes Bewegungs-Szenario
\end{itemize}


\section{Aufgetretene Schwierigkeiten}

\subsection{Laserscanner}

\begin{itemize}
	\item \glqq Unsichtbare Wand\grqq\ in ca. 4m Entfernung
	\begin{itemize}
		\item Scanner nicht exakt waagerecht angebracht
		\item nicht relevant, da Entfernung von 4m ausreichend
	\end{itemize}
	\item \glqq Tote Pixel\grqq
	\begin{itemize}
		\item Fehlmessungen werden mit Wert 0 und nicht MAX belegt
		\item Roboter steht immer im Hindernis -> keine Fortbewegung
		\item Optionale Zuschaltung der Module per XML
	\end{itemize}
\end{itemize}


\subsection{Stromversorgung}

\begin{itemize}
	\item Robotereigene Stromversorgung reicht nicht aus für Roboter + Kinect
	\item Netzgerät nötig
	\item Einschränkung der Reichweite
\end{itemize}

\subsection{Kinect}

\begin{itemize}
	\item verliert getrackten Menschen bei zu schneller Bewegung
	\item Verringerung der Geschwindigkeit (Rotation sowie Translation)
\end{itemize}


\subsection{Weitere Probleme}

\begin{itemize}
	\item Odometrie auf Teppichboden
	\item SW-Bugs
	\item \ldots
\end{itemize}
