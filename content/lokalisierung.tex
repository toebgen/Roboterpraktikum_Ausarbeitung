% #############################################################################
%															Hinderniskarte und Lokalisierung
% #############################################################################
\chapter{Hinderniskarte und Lokalisierung}
\label{lokalisierung_cha}
% ********************************************************************************
% 										Aufgabenstellung
% ********************************************************************************
\section{Aufgabenstellung}
\label{lokalisierung_aufgabenstellung_sec}
\authorsection{\editorandreas}
% -----------------------------------------------------------------------------
%													Lokalisierung
\subsection{Lokalisierung}
Die Aufgabe der Lokalisierung, im Kontext der Robotik, ist die Bestimmung der
Roboter Pose, welche die Position und Orientierung des Roboters in seiner Umgebung vollständig beschreibt.
 Um Bahnplanung und Steuerung zu ermöglichen muss die Pose des Roboters bekannt sein,
 damit sich dieser gefahrlos fortbewegen und sicher mit der Umwelt interagieren kann.
 Für einen Menschen ist die Lokalisierung selbstverständlich, für einen Roboter
 stellt sie allerdings eine Herausforderung dar.
 Aufgrund einer unbekannter Startposition oder von Messungenauigkeiten während
 der Fortbewegung, kann die Pose eines Roboters nur in den seltensten Fällen
 genau bestimmt werden.
  Von lokaler Lokalisierung wird gesprochen wenn die aktuelle Pose des Roboters
 in seiner Umwelt bekannt ist und bei einer Positionsänderung des Roboters
 fortlaufend aktualisiert wird um sie auf dem neusten Stand zu halten. Hier
 werden Schätzungen der Bewegung aufgrund der Odometrie Sensorik mit Hilfe von weiteren Daten zur Posenbestimmung korrigiert. Dieses Vorgehen soll eine Aufsummierung von Fehlern bei der
 Schätzung der Bewegung durch die Odometrie verhindern. Im Gegensatz dazu ist
 bei der globalen Lokalisierung die aktuelle Pose des Roboters in seiner Umwelt
 nicht bekannt und muss zum Beispiel anhand von Kartenmaterial geschätzt werden.
 Man sieht leicht ein, dass bei dieser Art der Lokalisierung die Initiale
 Schätzung einen beliebig großen Fehler aufweisen kann.
 Nach dieser Einführung sollte klar geworden sein das es notwendig
 ist eine Lokalisierung durchzuführen um es unserem Roboter zu
 ermöglichen sich fortzubewegen.
% -----------------------------------------------------------------------------
%													Hinderniskarte
\subsection{Hinderniskarte}
Abgesehen von der Lokalisierung gibt es noch einen weiteren Grund warum es für einen mobilen Roboter unumgänglich ist,
 dass eine interne Repräsentation seiner Umgebung existiert. Karten, die zur Lokalisierung eingesetzt werden,
 enthalten meist nicht alle statischen Hindernisse, da diese Karten zu einem gewissen Zeitpunkt erstellt wurden und
 in der Zwischenzeit wahrscheinlich neue Objekte hinzugekommen oder bereits Vorhandene versetzt worden sind.
 Außerdem enthalten solche Karten offensichtlich keine Hindernisse die sich bewegen können, wie beispielsweise Menschen.
 Ein Hindernis ist ganz allgemein ein Bereich der von dem Roboter nicht befahren werden kann. Eine Darstellung der Hindernisse
 relativ zur Roboterpose wird als Hinderniskarte bezeichnet. Eine solche Karte
 kann zum Beispiel mit Hilfe von Laser- oder Kameradaten erstellt werden.
 Eine Kombination des Materials mehrerer Sensoren ist normalerweise das Sinnvollste,
 da Laser beispielsweise keine negativen Hindernisse erkennen können.
 Als negative Hindernisse werden Bereiche bezeichnet die im Vergleich zur
 gegenwärtigen Position eine geringere Höhe aufweisen.
 Ein Beispiel für ein solches Hindernis ist eine Treppe in ein tiefer gelegenes
 Stockwerk. Das erstellen einer Hinderniskarte ist aus den oben genannten
 Gründen essentiell um eine Bahnplanung für einen mobilen Roboter
 durchführen zu können.
% -----------------------------------------------------------------------------
%													Virtual protective Field
\subsection{Virtual protective Field}
Die Aufgabe der Bahnplanung besteht darin auf Grundlage der Lokalisierung und der Hinderniskarte einen Weg zu einer
 bestimmten Position zu finden. Danach wird dieser Weg vom Roboter abgefahren. Allerdings kann es während dem Abfahren der gefundenen
 Route Aufgrund eines Hindernisses, dass sich in den Weg des Roboters bewegt bevor erneut geplant werden kann oder
 schlicht Aufgrund eines Fehlers in der Planung dazu kommen, dass der Roboter mit einem Hindernis kollidiert.
 Daher wird eine Möglichkeit benötigt auf einer tieferen Ebene wie der Bahnplanung das unbeabsichtigte kollidieren mit einem Hindernis
 zu verhindern. An dieser Stelle kommt das sogenannte „Virtual protective Field“ ins Spiel, von welchem ein Bereich um den Roboter auf
 Hindernisse überprüft und beim Auffinden eines Solchen ein Signal erzeugt wird, damit der Roboter gestoppt werden kann bevor er
 auf das Hindernis trifft. Dieser Bereich muss mit der Geschwindigkeit des
 Roboters wachsen bzw. schrumpfen so das sichergestellt ist, dass auf jeden
 Fall eine Kollision verhindert werden kann.
% ********************************************************************************
% 										Grundlagen Lokalisierung
% ********************************************************************************
\section{Grundlagen Lokalisierung}
\label{lokalisierung_grundlagen_sec}
\authorsection{\editordummy}
\todo[inline]{Lokalisierungs-Gruppe: Schreiben}
% -----------------------------------------------------------------------------
%													Lokalisierung
\subsection{Lokalisierung}
Nachdem wir im vorigen Abschnitt neben einer kurzen Erläuterung der
Materie hauptsächlich die Aufgaben und deren Ziele betrachtet haben,
 wollen wir hier eine detaillierte Betrachtung der Möglichkeiten
 und Verfahren zur Positionsbestimmung durchführen.
 
\subsubsection{Odometrie}
Ein einfacher Ansatz zur Positionsbestimmung ist die Verwendung
 von Schrittzählern, welche die Bewegung der Roboterräder oder
 -Beine misst. Über eine Umrechnungsfunktion (meist: lineare
 Abbildung) lässt sich aus der Anzahl der zurückgelegten Schritte
 eine Pose relativ zur Ausgangspose berechnen.

\begin{itemize}
  \item Vorteile:
  \begin{itemize}
    \item Relativ Stör-unanfällig unter Normalbedingungen\\ 
    (Ausnahmen: glatter, weicher, unebener Untergrund)
    \item Vernachlässigbarer Rechenaufwand
    \item Vernachlässigbarer Implementations-Aufwand
  \end{itemize}
  \item Nachteile:
  \begin{itemize}
    \item Offset-Fehler steigt linear mit dem Betrag der Schrittzahl
    \item Ursprungsposition muss bekannt sein (vorgegeben werden)
    \item Fehler abhängig vom Untergrund
    \item Bei auf Schlupf basierenden (omnidirektionalen)
     Plattformen Fehler höher
   \end{itemize}
\end{itemize}

\subsubsection{Kartenbasierte Lokalisierung}

Anhand von gegebenem Kartenmaterial kann aus einem Sensorbild
 der Umgebung die Position des Roboter geschätzt werden.
 Dazu werden zum Beispiel markante Punkte aus dem Sensorbild mit den
 Daten des Kartenmaterials abgeglichen. Im bestmöglichen Fall
 existiert nur eine mögliche Übereinstimmung, anhand derer der
 Roboter seine eigene Position bestimmen kann.

\begin{itemize}
  \item Vorteile:
  \begin{itemize}
    \item Keine Fehlerakkumulation über Zeitverlauf
    \item Keine Ursprungsposition notwendig
  \end{itemize}
  \item Nachteile:
  \begin{itemize}
    \item Störanfällig (spiegelnde Oberflächen, bewegende Objekte)
    \item Benötigt aktuelles Kartenmaterial
    \item hoher Rechenaufwand
    \item hoher Implementierungs-Aufwand
   \end{itemize}
\end{itemize}

\subsubsection{Landmarkbasierte Lokalisierung}

In einer bekannten Umgebung können Landmarks (QR-Codes, Bodenlinien,
 GPS-Satelliten) angebracht werden. Diese können durch Sensoren
 am Roboter wiedererkannt werden und ermöglichen die Lokalisierung
 z.B. mittels Triangulierung.

\begin{itemize}
  \item Vorteile:
  \begin{itemize}
    \item (durchschnittlicher) Fehler niedrig
    \item Zuverlässig (max. Fehler niedrig)
    \item Keine Fehlerakkumulation über Zeitverlauf
  \end{itemize}
  \item Nachteile:
  \begin{itemize}
    \item Landmarks notwendig
    \item Landmarks müssen von überall sichtbar sein
    \item Kartenmaterial u.U. notwendig
   \end{itemize}
\end{itemize}

Um die Nachteile der verschiedenen Lokalisierungsmethoden zu
 verringern werden in der Regel mindestens zwei Methoden miteinander
 kombiniert. Aufgrund der einfachen Implementierung und der geringen
 Störanfälligkeit ist meistens die Odometrie in Kombination mit
 anderen Systemen vorzufinden. Bei Verwendung mehrerer Methoden sind
 dementsprechend viele fehlerbehaftete Annahmen der aktuellen
 Roboterpose gegeben. Im Folgenden wollen wir zwei unterschiedliche
 Verfahren vorstellen, welche aus diesen Annahmen bestmöglich die
 aktuelle Roboterpose bestimmen.
 
\subsubsection{Kalman-Filter}
  Ein Verfahren, um den gesamten Fehler zu minimieren ist das
  Kalman-Filter.
\todo[inline]{Kurze Erläuterung / Einführung in das
Kalman-Filter}
\subsubsection{Partikel-Filter}
\todo[inline]{Kurze Erläuterung / Einführung in den
 Partikel Filter}
 
% ********************************************************************************
% 										Umsetzung
% ********************************************************************************
\section{Umsetzung}
\label{lokalisierung_umsetzung_sec}
\authorsection{\editordummy}
% -----------------------------------------------------------------------------
%													Lokalisierung
\subsection{Lokalisierung}
\todo[inline]{Geplanter Aufbau -> kurze Beschreibung der
 Vorhandenen und zur Umsetzung verwendbaren Gruppe und Module
 --- Grundlegende Idee für die Umsetzung
 --- Bild der Gruppe, erläutern der Funktion der Module}
 
 Vorhandene
 Implementierung der Lokalisierung anhand von Odometrie und Laserdaten im Odete-Projekt
Hardware Abstraction Layer (Hal) und (später) Scanner Abstraction Layer (Sal)
aus dem bereits bestehenden Segway Omni Projekt

Übernahme der MCA2-Struktur zur Lokalisierung von Odete

Adaption der Vorhanden Gruppen und Module zur Lokalisierung 
Anpassung von LLSP, HLSP, Hal \& Sal an das Segway Omni Projekt
Erweiterung auf beliebige Menge von Scannern
Einbindung eines PixelRemovers zur Wegnahme fehlerhafter Scanner-Pixel

% -----------------------------------------------------------------------------
%													Hinderniskarte
\subsection{Hinderniskarte}
 Im Gegensatz zur Lokalisierung gab es für die von uns gewünschte Lösung dieser
 Aufgabenstellung noch keine Gruppen oder Module von älteren Projekten. Aus
 diesem Grund war es notwendig im Vorfeld intensivere Überlegungen über die möglichen
 Umsetzungen und Repräsentationen anzustellen.
 Nach einigen Diskussionen entschieden wir uns für die Nutzung von
 Bildverarbeitungs"=Modulen aus SenseIVT \todoprivate{ref setzen} und einer
 Repräsentation der Hinderniskarte als Schwarzweiß-Bitmap. Der Grund für diese Wahl lag vor allem an der guten
 Visualisierung, die dadurch erhalten werden kann, außerdem wird eine einfache Anpassung und die Verwendung
 von Standard-Filtern ermöglicht. 
\missingfigure{Hinderniskarte}
%\begin{figure}[h]
%\center
%\includegraphics[]{}
%\caption{\label{fig:hinderniskarte} Hindernisskarte als Schwarzweiß-Bitmap}
%\end{figure}
 In Abbildung \ref{fig:hinderniskarte} ist ein Beispiel eines solches Bitmaps
 zu sehen. Mit der Semantik, dass weiße Areale Hindernisse darstellen und schwarze Bereiche entweder befahrbar
 oder von dieser Position aus nicht einsehbar sind. Diese von uns gewählte Abbildung der Werte
 auf die entsprechenden Bedeutungen ist allerdings beliebig und hat keinen besonderen Grund.
 Die Roboterposition ist der Mittelpunkt der Karte. Bevor Hindernisse in eine solche Karte eingezeichnet werden
 können müssen sie offensichtlich zuvor durch Sensoren detektiert worden sein. Es lag nahe die Laserdaten,
 welche auch zur Lokalisierung eingesetzt und bereits in geeigneter Form vorhanden waren zu nutzen.
 Da ein Roboter in der Realität eine größere Fläche als ein Punkt aufweist entschieden wir uns bereits an dieser
 Stelle zu einer Expansion der Hindernisse auf Robotergröße, damit solche Berechnungen an anderer Stelle vermieden
 werden können.
\missingfigure{MCA2 Obstacle Extractor}
%\begin{figure}[h]
%\center
%\includegraphics[]{}
%\caption{\label{fig:obstacleExtractor} Hindernisskarte als Schwarzweiß-Bitmap}
%\end{figure}
 In Abbildung \ref{obstacleExtractor} ist die die Gruppe (\lstinline{Obstacle Extractor}),
 durch welche die gewünschte Funktionalität implementiert wurde zu sehen.
 Im Folgenden wollen wir auch hier die einzelnen Module kurz vorstellen:

\todo[inline]{Erläutern der Funktion der Module}

\begin{description}
\item[from to gedöns]
\item[bla bla]
\item[bla bla bla]
\item[aaa]
\item[aaa]
\end{description}



% -----------------------------------------------------------------------------
%													Virtual protective Field
\subsection{Virtual protective Field}
\todo[inline]{Geplanter Aufbau -> kurze Beschreibung der
 Vorhandenen und zur Umsetzung verwendbaren Gruppe und Module
 --- Grundlegende Idee für die Umsetzung
 --- Bild der Gruppe, erläutern der Funktion der Module}
 
 Virtual Protective Field lediglich für Differentialantrieb implementiert
 
    Erstellen eines neuen Virtual Protective Fields aufgrund der Fähigkeit zur omnidirektionalen Bewegung
        Ausnutzen der bereits bestehenden Hindernisskarte
            Verwendung von Bildverarbeitungs-Modulen (SenseIVT)

Implementierung als Schwarz-Weiß-Bitmap

        3 unterschiedliche Virtual Protective Fields
            Sideward
            Forward
            Combined
        Dynamische Ausdehnung proportional zu (Geschwindigkeit)ins quadrat
        AND-Operation zwischen Hindernis- und Feldpixeln
        Ausgabe von Controller Output bei erkannten Hindernissen
