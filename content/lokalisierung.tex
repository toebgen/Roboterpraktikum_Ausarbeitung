% #############################################################################
%															Hinderniskarte und Lokalisierung
% #############################################################################
\chapter{Hinderniskarte und Lokalisierung}
\label{lokalisierung_cha}
% ********************************************************************************
% 										Aufgabenstellung
% ********************************************************************************
\section{Aufgabenstellung}
\label{lokalisierung_aufgabenstellung_sec}
\authorsection{\editorandreas}
% -----------------------------------------------------------------------------
%													Lokalisierung
\subsection{Lokalisierung}
Die Aufgabe der Lokalisierung, im Kontext der Robotik, ist die Bestimmung der
Roboter Pose, welche die Position und Orientierung des Roboters in seiner Umgebung vollständig beschreibt.
 Um Bahnplanung und Steuerung zu ermöglichen muss die Pose des Roboters bekannt sein,
 damit sich dieser gefahrlos fortbewegen und sicher mit der Umwelt interagieren kann.
 Für einen Menschen ist die Lokalisierung selbstverständlich, für einen Roboter
 stellt sie allerdings eine Herausforderung dar.
 Aufgrund einer unbekannter Startposition oder von Messungenauigkeiten während
 der Fortbewegung kann die Pose eines Roboters nur in den seltensten Fällen genau bestimmt werden. Für eine Lokalisierung wird weiterhin eine Karte der Umgebung benötigt,
 da ohne diese die Pose nicht bestimmte werden kann. Von lokaler Lokalisierung wird gesprochen wenn die aktuelle Pose des Roboters
 in seiner Umwelt bekannt ist und sie bei einer Bewegung des Roboters
 fortlaufend aktualisiert wird um sie auf dem neusten Stand zu halten. Hier
 werden Schätzungen der Bewegung aufgrund der Odometrie Sensorik mit Hilfe von weiteren Daten zur Posenbestimmung korrigiert. Dieses Vorgehen soll eine Aufsummierung von Fehlern bei der
 Schätzung der Bewegung durch die Odometrie verhindern. Im Gegensatz dazu ist
 bei der globalen Lokalisierung die aktuelle Pose des Roboters in seiner Umwelt
 nicht bekannt und muss zum Beispiel anhand von Kartenmaterial geschätzt werden.
 Man sieht leicht ein, dass bei dieser Art der Lokalisierung die Initiale
 Schätzung einen beliebig großen Fehler aufweisen kann. Wie wir gesehen haben
 ist es also notwendig eine Lokalisierung durchzuführen um es unserem Roboter zu
 ermöglichen sich fortzubewegen.
% -----------------------------------------------------------------------------
%													Hinderniskarte
\subsection{Hinderniskarte}
Abgesehen von der Lokalisierung gibt es noch einen weiteren Grund warum es für einen mobilen Roboter unumgänglich ist,
 dass eine interne Repräsentation seiner Umgebung existiert. Karten, die zur Lokalisierung eingesetzt werden,
 enthalten meist nicht alle statischen Hindernisse, da diese Karten zu einem gewissen Zeitpunkt erstellt wurden und
 in der Zwischenzeit wahrscheinlich neue Objekte hinzugekommen oder bereits Vorhandene versetzt worden sind.
 Außerdem enthalten solche Karten offensichtlich keine Hindernisse die sich bewegen können, wie beispielsweise Menschen.
 Ein Hindernis ist ganz allgemein ein Bereich der von dem Roboter nicht befahren werden kann. Eine Darstellung der Hindernisse
 relativ zur Roboterpose wird als Hinderniskarte bezeichnet. Eine solche Karte
 kann zum Beispiel mit Hilfe von Laser- oder Kameradaten erstellt werden.
 Eine Kombination des Materials mehrerer Sensoren ist normalerweise das Sinnvollste,
 da Laser beispielsweise keine negativen Hindernisse erkennen können.
 Als negative Hindernisse werden Bereiche bezeichnet die im Vergleich zur
 gegenwärtigen Position eine geringere Höhe aufweisen.
 Ein Beispiel für ein solches Hindernis ist eine Treppe in ein tiefer gelegenes
 Stockwerk. Das erstellen einer Hinderniskarte ist aus den oben genannten
 Gründen essentiell um eine Bahnplanung für einen mobilen Roboter durchzuführen.
% -----------------------------------------------------------------------------
%													Virtual protective Field
\subsection{Virtual protective Field}
Die Aufgabe der Bahnplanung besteht darin auf Grundlage der Lokalisierung und der Hinderniskarte einen Weg zu einer
 bestimmten Position zu finden. Danach wird dieser Weg vom Roboter abgefahren. Allerdings kann es während dem Abfahren der gefundenen
 Route Aufgrund eines Hindernisses, dass sich in den Weg des Roboters bewegt bevor erneut geplant werden kann oder
 schlicht Aufgrund eines Fehlers in der Planung dazu kommen, dass der Roboter mit einem Hindernis kollidiert.
 Daher wird eine Möglichkeit benötigt auf einer tieferen Ebene wie der Bahnplanung das unbeabsichtigte kollidieren mit einem Hindernis
 zu verhindern. An dieser Stelle kommt das sogenannte „Virtual protective Field“ ins Spiel, von welchem ein Bereich um den Roboter auf
 Hindernisse überprüft und beim Auffinden eines Solchen ein Signal erzeugt wird, damit der Roboter gestoppt werden kann bevor er
 auf das Hindernis trifft. Dieser Bereich muss mit der Geschwindigkeit des
 Roboters wachsen bzw. schrumpfen so das sichergestellt ist, dass auf jeden
 Fall eine Kollision verhindert werden kann.
% ********************************************************************************
% 										Grundlagen Lokalisierung
% ********************************************************************************
\section{Grundlagen Hinderniskarte und Lokalisierung}
\label{lokalisierung_grundlagen_sec}
\authorsection{\editordummy}
\todo[inline]{Lokalisierungs-Gruppe: Schreiben}
% -----------------------------------------------------------------------------
%													Lokalisierung
\subsection{Lokalisierung}
 Odometrie

Ein einfacher Ansatz zur Positionsbestimmung ist die Verwendung von Schrittzählern, welche die Bewegung der Roboterräder oder -Beine misst. Über eine Umrechnungsfunktion (meist: lineare Abbildung) lässt sich aus der Anzahl der zurückgelegten Schritte eine Pose relativ zur Ausgangspose berechnen.
Vorteile

    Relativ Stör-unanfällig unter Normalbedingungen (Ausnahmen: glatter, weicher, unebener Untergrund)
    Vernachlässigbarer Rechenaufwand
    Vernachlässigbarer Implementations-Aufwand 

Nachteile ¶

    Offset-Fehler steigt linear mit dem Betrag der Schrittzahl
    Ursprungsposition muss bekannt sein (vorgegeben werden)
    Fehler abhängig vom Untergrund
    Bei auf Schlupf basierenden (omnidirektionalen) Plattformen Fehler höher 
    
Kartenbasierte Lokalisierung

Anhand von gegebenem Kartenmaterial kann aus einem Sensorbild der Umgebung die Position des Roboter geschätzt werden. Dazu werden zum Beispiel markante Punkte aus dem Sensorbild mit den Daten des Kartenmaterials abgeglichen. Im bestmöglichen Fall existiert nur eine mögliche Übereinstimmung, anhand derer der Roboter seine eigene Position bestimmen kann.
Vorteile

    Keine Fehlerakkumulation über Zeitverlauf
    Keine Ursprungsposition notwendig 

Nachteile

    Störanfällig (spiegelnde Oberflächen, bewegende Objekte)
    Benötigt aktuelles Kartenmaterial
    hoher Rechenaufwand
    hoher Implementierungs-Aufwand 
    
Landmarkbasierte Lokalisierung

In einer bekannten Umgebung können Landmarks (QR-Codes, Bodenlinien, GPS-Satelliten) angebracht werden. Diese können durch Sensoren am Roboter wiedererkannt werden und ermöglichen die Lokalisierung z.B. mittels Triangulierung.
Vorteile

    Fehler niedrig (durchschnittlicher Fehler niedrig)
    Keine Fehlerakkumulation über Zeitverlauf
    Zuverlässig (max. Fehler niedrig) 

Nachteile

    Landmarks notwendig
    Landmarks müssen von überall sichtbar sein
    Kartenmaterial u.U. notwendig 
    
Kombination der Messverfahren

Um die Nachteile der verschiedenen Lokalisierungsmethoden zu verringern werden in der Regel mindestens zwei Methoden miteinander kombiniert. Aufgrund der einfachen Implementierung und der geringen Störanfälligkeit ist meistens die Odometrie in Kombination mit anderen Systemen vorzufinden. Bei Verwendung mehrerer Methoden sind dementsprechend viele fehlerbehaftete Annahmen der aktuellen Roboterpose gegeben. Aus diesen Annahmen wird bestmöglichst die aktuelle Roboterpose bestimmt. Ein Verfahren, um den gesamten Fehler zu minimieren ist das Kalman-Filter. 
 Das Kalman-Filter

    TODO: Kurze Erläuterung / Einführung in das Kalman-Filter

% -----------------------------------------------------------------------------
%													Hinderniskarte
\subsection{Hinderniskarte}

% -----------------------------------------------------------------------------
%													Virtual protective Field
\subsection{Virtual protective Field}

% ********************************************************************************
% 										Umsetzung
% ********************************************************************************
\section{Umsetzung}
\label{lokalisierung_umsetzung_sec}
\authorsection{\editordummy}

% -----------------------------------------------------------------------------
%													Lokalisierung
\subsection{Lokalisierung}
Vorhandene Implementierung der Lokalisierung anhand von Odometrie und Laserdaten
im Odete-Projekt
Hardware Abstraction Layer (Hal) und (später) Scanner Abstraction Layer (Sal)
aus dem bereits bestehenden Segway Omni Projekt

Übernahme der MCA2-Struktur zur Lokalisierung von Odete

Adaption der Vorhanden Gruppen und Module zur Lokalisierung 
Anpassung von LLSP, HLSP, Hal \& Sal an das Segway Omni Projekt
Erweiterung auf beliebige Menge von Scannern
Einbindung eines PixelRemovers zur Wegnahme fehlerhafter Scanner-Pixel

% -----------------------------------------------------------------------------
%													Hinderniskarte
\subsection{Hinderniskarte}
Keine Hinderniskarte vorhanden

Erzeugung der Hindernisskarte aus den Laserdaten
Verwendung von Bildverarbeitungs-Modulen (SenseIVT)
            Gute Visualisierung
            Einfache Anpassung
            Verwendung von Standard-Filtern
         Repräsentation als Schwarz-Weiß-Bitmap
        Hindernisse auf den Radius des Roboters aufgebläht
        Anschließende Schließen Operation
% -----------------------------------------------------------------------------
%													Virtual protective Field
\subsection{Virtual protective Field}
Virtual Protective Field lediglich für Differentialantrieb implementiert

    Erstellen eines neuen Virtual Protective Fields aufgrund der Fähigkeit zur omnidirektionalen Bewegung
        Ausnutzen der bereits bestehenden Hindernisskarte
            Verwendung von Bildverarbeitungs-Modulen (SenseIVT)

Implementierung als Schwarz-Weiß-Bitmap

        3 unterschiedliche Virtual Protective Fields
            Sideward
            Forward
            Combined
        Dynamische Ausdehnung proportional zu (Geschwindigkeit)ins quadrat
        AND-Operation zwischen Hindernis- und Feldpixeln
        Ausgabe von Controller Output bei erkannten Hindernissen
% -----------------------------------------------------------------------------
%													Simulation
\subsection{Simulation}
