\chapter{Einleitung}
\label{einleitung_cha}
\todo[inline]{Author hinzufügen}
\authorsection{\editordummy}
Das aktuelle Interesse in der Forschung zur Robotik und insbesondere der Servicerobotik
 ist Robotersysteme flexibler zu gestalten und für breitere Anwendungsgebiete, zum Beispiel
 statt eines reinen Staubsaugroboters einen Haushaltsroboters zu verwenden.
 Im Fokus steht außerdem, wie auch in der Informationstechnologie eine kostengünstige Verbesserung des Mensch-Maschine-Interface.
\todo[inline]{Quelle kann nicht identifiziert werden}

Die Forschungsfelder in Industrie- und Servicerobotik nähern sich dabei im Laufe der Jahre immer weiter an.
 So müssen Industrieroboter in Zusammenarbeit mit dem Menschen ihre Aufgaben verrichten und Serviceroboter
 in Zukunft auch komplexe Handhabungsaufgaben lösen. Die Entwicklung findet daher besonders im Bereich der
 Informationstechnologie und der Sensorik statt \citep{Michael2010}.
 
In den letzten Jahrzehnten hat sich die Bildverarbeitung rasant weiterentwickelt.
 Zur meist funktionsrelevanten Interaktion eines Serviceroboters mit der Umwelt muss
 ein Roboter aufgrund der Vielfalt möglicher Umgebungen adaptiv in der Lage sein reichhaltige
 Informationen darüber zu gewinnen. Unstrukturierte Umgebungen lassen sich nicht vollständig modellieren.
 Roboter müssen somit durch Interaktion mit ihrer Umwelt lernen, sich orientieren und auf Menschen bzw. Informationen reagieren.
 Besonders anspruchsvoll sind dabei Perzeption und Verarbeitung von Signalen.

\section{Motivation und Problemstellung}
\label{motivation_sec}
\todo[inline]{Author hinzufügen}
\authorsection{\editordummy}

In vielen Fällen müssen sich Serviceroboter in unstrukturierten Umgebungen zurechtfinden in denen es nicht möglich ist diese durch
 (zum Beispiel markerbasierte) spezielle Kennzeichnung der Objekte mit Informationen anzureichern \citep{sturm10rss-workshop}.
 Auch Industrieroboter sollen in Zukunft enger mit (unberechenbaren) menschlichen Kooperationspartnern zusammenarbeiten.
 Kamerabilder stellen in diesem Kontext eine reichhaltige und kostengünstige Informationsquelle dar und können über Sensorfusion bzw. Musterprojektion zusätzlich
 zu 3D-Tiefenbildern erweitert werden. 

Das Identifizieren und Verfolgen geometrischer Objekte in (Tiefen-) Bildern ist ein weit entwickeltes Feld mit hoher Reife und etablierten Methoden.
 Die große Herausforderung ist die semantische Interpretation dieser Daten. Besonders das zuverlässige Gewinnen funktionsrelevanter Informationen,
 wie zum Beispiel interaktiver Charakteristika, kinematischer, werkstofftechnischer und  Oberflächen-Beschaffenheit aus (Tiefen-) Bildern stellt sich
 als problematisch dar. In unstrukturierten Umgebungen versagen andere Sensorentypen allein jedoch oder deren Anwendung ist bisher aufgrund des Preises sinnlos.
 
Aktuell werden verschiedene Ansätze verfolgt: Zum einen generiert und erfasst man weitere Sensorinformationen durch Interaktion eines Roboters mit der Umgebung,
 zum anderen werden dem Roboter weitreichende erwartete Zusammenhänge von Informationen bereitgestellt. Entwickler können bei diesen Aufgaben auf weit entwickelte Roboterplattformen
 zurückgreifen, wie den PR2 der Firma willow garage und wie in diesem Praktikum
 \gls{hollie} des \gls{fzi} Karlsruhe. Willow garage stellt auch umfangreiche
 Software zur (Tiefen-) Bildverarbeitung und Robotersteuerung zur Verfügung,
 wie \gls{opencv}, \gls{pcl} und \gls{ros}.

\section{Stand der Technik}
\label{stand_der_technik_sec}
\authorsection{\editordummy}
\todo[inline]{Kinect-Gruppe: Schreiben}

