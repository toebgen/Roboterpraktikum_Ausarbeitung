
% alle Abkürzungen, die in der Arbeit verwendet werden
%
% Paket 'glossaries' , ein Eintrag sieht so aus:
% 	Bemerkung:
%			- name kann auch ein Symbol sein ($\pi$)
%		Verwendung:	(für glossaries UND acronyms gleich!)
%			- \gls{label}
%			- \glspl{label}			(Plural)
%			- \Gls{label}			(groß geschrieben)
%			- \Glspl{label}
%			- \glsentrytext{label}	(zur Verwendung in Überschriften etc.)
% Acronyme:
%			\newacronym{label}{SVM}{Support Vector Machine}
%		Bemerkung:
%			- \glsreset{label}		(dann wird ab hier nochmal das ausgeschriebene angezeigt)
%			- \glsresetall

% \newglossaryentry{sdk}{name={Software Development Kit},text={Software Development Kit},
% description={Eine Schnittstelle um eigene Software an bestehende Programme, Applikationen, Betriebssysteme etc. anzubinden \cite{sdkComputerlexikon}}}

\newacronym{fzi}{FZI}{Forschungszentrum Informatik}
\newacronym{hollie}{HoLLiE}{House of Living Labs intelligent Escort}

\newglossaryentry{Konfigurationsraum}{name={Konfigurationsraum},
description={Ist der Raum alle möglichen Gelenkwinkelkonfigurationen des Roboters}}

\newglossaryentry{Voronoi}{name={Voronoi},
description={Verfahren zur zerlegung des Raumes in Regionen die von vorgegeben Punkte bestimmt werden. Jede Region wird durch genau ein Zentrum bestimmt und umfasst alle Punkte des Raumes, die in Bezug zur euklidischen Metrik näher an dem Zentrum der Region liegen, als an jedem anderen Zentrum}}

\newglossaryentry{Sichtgraphen}{name={Sichtgraphen},
description={Verfahren zur Konstruktion von Roadmaps. Man verbindet jedes Paar von Eckpunkten auf dem Rand von kollisionsfreien Konfigurationsraum durch ein gerades Liniensegment, wenn das Segment kein Hindernis schneidet}}

\newglossaryentry{Zellenzerlegung}{name={Zellenzerlegung},
description={Bei exakte Zellenzerlegung mahlt man horizontale Linien von jeden Eckpunkt aus bis man ein Hindernis trifft um Zellen zu schaffen. Bei approximative Verfahren werden Rechtecke rekursiv zerlegt bis alle Rechtecke entweder in Freiraum oder in ein Hinderniss liegen}}

\newglossaryentry{PRM}{name={PRM},
description={Probabilistische Roadmaps erzeugen zufällig Kollisionsfreie Punkte, je nach Sampling Strategie, mit dem man eine Graphen aufspannt, die Start und Zielknoten hinzufügt und danach mit einen lokalen Planer auf dem Graphen sucht}}

