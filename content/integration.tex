
\chapter{Integration}
\label{integration_cha}
\authorsection{\editorjulian}
\todo[inline]{Julian: Schreiben}

\section{Aufgabenstellung}
\label{aufgabenstellung_integration_sec}

\begin{itemize}
\item Überblick was alles unter Integration fällt
\item Zusammenführen der einzelnen Teilprojekte
\item Zustandsmaschinen zur Steuerung/Beobachtung der einzelnen Teile
\item Planer um auf Umwelteinflüsse zu reagieren
\item Userinterface zur Steuerung des Roboters
\end{itemize}

Die Aufgaben im Bereich der Integration lassen sich in die folgenden Unterpunkte einordnen:

\begin{description}
\item[Code Zusammenführen]
Um das Auseinanderdriften der einzelnen Teilprojekte zu vermeiden ist ein regelmäßiges Zusammenführen mittels Git notwendig. Zusätzlich ist die Integration von Code der außerhalb des SegwayOmni-Projektes entsteht, wie etwa der Kinect Anteil, notwendig. Weiterhin muss die Kompatibilität der Schnittstellen zwischen den Teilprojekten gepflegt werden.

\item[Benutzerschnittstelle]
Zur Kontrolle des Roboters durch Benutzer muss eine Benutzerschnittstelle implementiert werden die es erlaubt die einzelnen Funktionen der Software zu aktivieren.

\item[Reaktion auf Zustandsänderungen]
Der vorherrschende Umweltzustand muss erkannt und eine entsprechende Reaktion ausgeführt werden.
\end{description}




\section{Grundlagen der Planung}
\label{grundlagen_integration_sec}

\begin{itemize}
\item Generell Planen: sense -> plan -> act (loop)
\item Unterschiedliche Problemstellungen
\item Mögliche Ansätze
\item State of the Art
\end{itemize}



\section{Umsetzung}
\label{umsetzung_integration_sec}

\subsection{Architektur}
\label{integration_architektur_sec}

\begin{itemize}
\item Highlevel Architektur, insbesondere Zusammenspiel Planer, Bahnplanung, Kinect
\item Aufteilung in sense und control
\item Bild!
\end{itemize}

\subsection{Planer}
\label{planer_integration_sec}

\begin{itemize}
\item Einordnung des konkreten Problems mit Attributen (siehe RobI F20):
\begin{itemize}
\item Determinismus
\item Endlicher Zustandsraum
\item Diskreter Zustandsraum
\item Statischer Zustandsraum
\end{itemize}
\item Umsetzung mit Zustandsmaschine
\item Sense Plan Act loop konkret
\end{itemize}

\subsection{Benutzerschnittstelle}
\label{benutzerschnittstelle_integration_cha}

\begin{itemize}
\item Ausführbare Handlungen
\item Steuerung über Gesten
\item Ausführen der entsprechenden Handlung
\item Steuerung der einzelnen Statemachines
\end{itemize}

\section{Probleme}
\label{probleme_integration_sec}

\begin{itemize}
\item Definition der Schnittstellen
\item Definition der Funktionen des Roboters
\end{itemize}
