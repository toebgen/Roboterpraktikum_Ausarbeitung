
\chapter{Fazit}

\section{Erreichte Ziele}
\authorsection{Steuerungs-Gruppe}
\todo[inline]{Steuerungs-Gruppe: Schreiben}

\begin{itemize}
  \item Bedienung von HoLLiE mittels Gesten
	\begin{itemize}
	\item Gestenerkennung durch Kinect
	\end{itemize}
  \item Lokalisierung des Roboters in gegebener Karte
  \item Verfolgung des Menschen
	\begin{itemize}
	  \item Tracking des Menschen (Position und Geschwindigkeit)
	  \item Bahnplanung von Roboter zu Mensch
	  \item Speicherung des Pfads
	\end{itemize}
  \item Abfahren des gespeicherten Pfads
	\begin{itemize}
	  \item Rückkehr zum Startpunkt
	  \item Erneutes Abfahren (vorwärts)
	\end{itemize}
  \item Hindernis-Umfahrung
	\begin{itemize}
	  \item Bahnplanung
	  \item Protective-Field
	\end{itemize}
\end{itemize}

\section{Ausblick}
\todo[inline]{Author hinzufügen}
\authorsection{Kinect-Gruppe}
Serviceroboter sollen sich von spezialisierten Maschinen hin zu universell arbeitenden Helfern entwickeln.
 Dies umfasst Entwicklungen in der Perzeption, Interpretation, Handhabung etc. Im Roboterpraktikum konnten wir einen guten Eindruck von den ersten beiden Gebieten gewinnen.
 Im alltäglichen Betrieb bereits ausgereift ist die Orientierung mittels
 Laserscanner. Auch die Bildverarbeitung ist im 3D-Bereich mit Gesichtserkennung
 und vieler weiterer Verarbeitungsverfahren weit entwickelt. Im Feld von alltagstauglicher 3D-Kameratechnologie sind jedoch viele Verfahren noch im Forschungsstatus
 und noch nicht breit anwendbar. Besonders im Bereich von Servicerobotern, die sich in einer Vielzahl von Umgebungen zurechtfinden sollen stellt dies aktuell ein Problem dar,
 da dem Roboter nicht a priori alle Informationen über seine Umwelt bereitgestellt werden können.
 
Sobald diese Technologien ausgereifter sind wird dich die Umgebungswahrnehmung von Robotern weit verbessern.
 Zum jetzigen Zeitpunkt ist aber besonders erschwingliche 3D-Kameratechnologie im Anwendungsbereich noch auf Spezialanwendungen wie Entertainmentprodukte beschrängt.
 Auch die geringe Genauigkeit dieser Sensoren führt dazu, dass die gewonnenen Informationen nur mäßig präzise sind.

\section{Reflexion}
\authorsection{\editorandreas}
% -----------------------------------------------------------------------------
%													Organisatorische Herausforderungen
\subsection{Organisatorische Herausforderungen}
Diese Arbeit entstand im Rahmen eines Projektpraktikums, im Gegensatz
 zu einem reinen Praktikum oder einem Seminar lag daher auch
 die Planung und die Organisation im Aufgabengebiet der am Projekt
 beteiligten. Von Anfang an hatten wir, die Teilnehmer an diesem
 Projekt, große Freiheiten was die Ausgestaltung der letztendlichen
 Ziele betraf, auch waren wir relativ uneingeschränkt bei der Planung
 der Deadlines dieser Ziele. Die erste organisatorische
 Herausforderung des Praktikums bestand darin realistisch umsetzbare
 Ziele zu finden, allerdings hatten die Meisten von uns in diesem
 Gebiet keinerlei Erfahrungen. Anfangs gab es relativ viele Vorschläge
 aber schlussendlich entschieden wir uns, auf anraten der Betreuer,
 nur wenige in die Planung aufzunehmen. Hier sollte noch erwähnt
 werden, dass selbst mit einer großen Erfahrung im Gebiet der
 Planung und Organisation, ohne vorherige Einarbeitung in die Materie
 und die vorhanden Primitiven, eine genaue Abschätzungen sehr
 schwierig ist. Eine mehrmonatige Einarbeitungszeit wie in der Praxis
 üblich stand uns aufgrund der Art der Veranstaltung
 verständlicherweise nicht zur Verfügung.
 Das Aufteilen der anstehenden Aufgaben bereitete keine Probleme,
 allerdings war die Kommunikation zwischen den einzelnen Gruppen
 anfangs etwas spärlich. Jedoch konnten wir gegen Ende des
 Praktikums dieses Problem und die daraus resultierenden
 Schwierigkeiten erfolgreich ausräumen. Die unterschiedlichen
 Verpflichtungen der Teilnehmer führten dazu, dass auch das
 Einhalten interner Deadlines eine gewisse Herausforderung
 darstellte. Weiterhin gab es durch kontinuierliche Änderungen
 an der Softwareumgebung einige Probleme. Eines der größten Probleme
 in diesem Zusammenhang betraf die Gruppe, die für die
 Gestenerkennung und das Tracking verantwortlich war.
 Im Verlauf des Praktikums stellte sich leider heraus,
 dass das geplante Framework für die Kinekt, in dessen Integration
 schon einige Arbeit investiert worden war, nicht vollständig
 mit realistischem Aufwand in die Softwareumgebung integriert werden
 kann. Dies führte dazu, dass die komplette Planung verworfen
 werden musste. Die effiziente Nutzung der begrenzten Ressourcen
 stellte eine weitere organisatorische Herausforderung dar.
 Zu diesen Ressourcen gehörten zum einen, die begrenzten Arbeitsplätze
 welche ein paralleles arbeiten erschwerten und zum Anderen die
 begrenzte Zeit, in der der Roboter für Testzwecke zur Verfügung
 stand.
% -----------------------------------------------------------------------------
%													Fachliche Herausforderungen
\subsection{Fachliche Herausforderungen}
Nach den organisatorischen wollen wir nun die fachlichen
 Herausforderungen, vor die wir im Verlauf des Projekts gestellt
 wurden betrachten. Die größte war zweifellos die Einarbeitung in
 MCA2 , insbesondere in die von uns benötigten und bereits bestehenden
 Gruppen und Module. Die spärlichen Kommentare des Quelltextes
 beziehungsweise das Fehlen einer Dokumentation an einigen Stellen
 machten ein Verstehen der Objekte ohne die Hilfe einer Personen, die
 diese entweder erstellt oder gut in sie eingearbeitet war sehr
 schwierig. Offensichtlich ist es nicht mögliche sinnvolle Parameter
 zu wählen oder benötigte Anpassungen vorzunehmen wenn man den
 Quelltext nicht nachvollziehen kann. Die Programmierkenntnisse
 innerhalb der Gruppe der Teilnehmer reichten außerdem von so gut
 wie überhaupt keinen Vorkenntnissen bis zu Erfahrungen mit einigen
 relativ großen Projekten. Diese Unterschiede machten eine
 Koordinierung beziehungsweise eine noch intensiver Einarbeitung
 notwendig.
% -----------------------------------------------------------------------------
%													Learned Lessons
\subsection{Learned Lessons}
Zum Abschluss dieser Ausarbeitung wollen wir hier noch auf die von
 uns gewonnen Erkenntnisse eingehen. Durch eine an die Praxis
 angelehnte Durchführung des Projekts konnten wir einige Einsichten
 in den Ablauf von Projekten außerhalb des theoretischen
 Universitätsumfeldes erlangen. Die Unterschiede und Gemeinsamkeiten,
 die dadurch ersichtlich wurden können mit Sicherheit als wissenswert
 bezeichnet werden. Außerdem konnten erste Erfahrungen im Gebiet der
 Organisation und Planung von größeren Projekten gesammelt
 werden. Auch wurde uns realistisch die Bedeutung einer sorgfältigen
 Schnittstellendefinition sowie einer regen Kommunikation unterhalb
 der am Projekt beteiligten Parteien durch die Bearbeitung der
 einzelnen Aufgaben in unterschiedlichen Gruppen, die jedoch auf die
 Ergebnisse der jeweils anderen angewiesen waren, vor Augen geführt.
 In diesem Zusammenhang wurde auch deutlich warum die ausführliche
 und verständliche Dokumentation des erstellten Programmcodes von
 erheblicher Bedeutung ist. Außerhalb eines Praktikums an einem
 entsprechenden Institut stehen einem Studenten normalerweise keine
 komplexen mobilen Robotersysteme dieses Ranges zur Verfügung,
 daher war die Möglichkeit ein solches nutzen zu können wohl einer
 der positivsten Aspekte des Praktikums. Die praktischen Tests,
 der von uns erstellten Algorithmen konnten uns einen ersten Eindruck
 vermitteln wie viel Zeit aufgrund des eklatanten Kontrastes
 zwischen Simulation und echter Hardware benötigt und eingeplant
 werden muss.
